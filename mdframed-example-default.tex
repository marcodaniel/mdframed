%Documenation of the package mdframed
%%$Id: mdframed-example-default.tex 237 2011-11-27 12:52:31Z marco $
\setcounter{errorcontextlines}{999}
\documentclass[parskip=false,english,11pt]{ltxmdf}
\svnInfo $Id: mdframed-example-default.tex 237 2011-11-27 12:52:31Z marco $
\usepackage{babel}
\usepackage[utf8]{inputenc}
\usepackage[T1]{fontenc}
\usepackage[T1,altbullet]{lucidabr}
\usepackage[scaled=0.82]{beramono}  
\usepackage{showframe}
\usepackage[]{showexpl}
\lstset{style=lstmdframed,explpreset={pos=b,rframe={}},}

\newcommand\Loadedframemethod{default}
\usepackage[framemethod=\Loadedframemethod]{mdframed}

\title{The \mdname package}
\subtitle{Examples for \mdoption{framemethod=\Loadedframemethod}}
\author{\href{mailto:marco.daniel@mada-nada.de}{Marco Daniel}}
\version{\mdversion}
\date{\svnToday}
\introduction{In this document I collect various examples for \mdoption{framemethod=\Loadedframemethod}. 
Some presented examples are more or less exorbitant.}


\mdfsetup{skipabove=\topskip,skipbelow=\topskip}
\newrobustcmd\ExampleText{%
        An \textit{inhomogeneous linear} differential equation has the form
         \begin{align}
            L[v ] = f,
         \end{align}
        where $L$ is a linear differential operator, $v$ is
        the dependent variable, and $f$ is a given non-zero 
        function of the independent variables alone.
}


\newcounter{examplecount}
\setcounter{examplecount}{0}
\renewcommand\thesubsection{}
\newcommand\Examplesec[1]{%
\stepcounter{examplecount}%
\subsection{Example~\arabic{examplecount}~--~#1\relax}%
}

\begin{document}
\maketitle
\section{Loading}
In the preamble only the package \mdname width the option \mdoption{framemethod=\Loadedframemethod} is loaded. All other modifications will be done by \mdcommand{mdfdefinestyle} or \mdcommand{mdfsetup}.

{\large\color{red!50!black}
\NOTE Every \mdcommand{global} inside the examples is necessary to work with the package \mdpack{showexpl}.}

\section{Examples}
All examples have the following settings:

\begin{mdexample}
\mdfsetup{skipabove=\topskip,skipbelow=\topskip}
\newrobustcmd\ExampleText{%
An \textit{inhomogeneous linear} differential equation
has the form
\begin{align}
L[v ] = f,
\end{align}
where $L$ is a linear differential operator, $v$ is
the dependent variable, and $f$ is a given non-zero 
function of the independent variables alone.
}
\end{mdexample}
\clearpage
\Examplesec{very simple}
\begin{LTXexample}
\global\mdfdefinestyle{exampledefault}{%
     linecolor=red,linewidth=3pt,%
     leftmargin=1cm,rightmargin=1cm
}
\begin{mdframed}[style=exampledefault]
\ExampleText
\end{mdframed}
\end{LTXexample}


\Examplesec{hidden line + frame title}
\begin{LTXexample}
\global\mdfapptodefinestyle{exampledefault}{%
 topline=false,rightline=true,bottomline=false}
\begin{mdframed}[style=exampledefault,frametitle={Inhomogeneous linear}]
\ExampleText
\end{mdframed}
\end{LTXexample}
\clearpage

\Examplesec{colored frame title}
\begin{LTXexample}
\renewcommand\mdframedtitleenv[1]{%
                 \colorbox{green}{%
                 \parbox{\dimexpr\linewidth-6pt\relax}%6pt=linewidth 
                 {\centering\bfseries #1}}%
                 \par\kern.5\baselineskip\noindent%
        }
\global\mdfapptodefinestyle{exampledefault}{%
   rightline=true,innerleftmargin=0,innerrightmargin=0}
\begin{mdframed}[style=exampledefault,frametitle={Inhomogeneous linear}]
\ExampleText
\end{mdframed}
\end{LTXexample}



\Examplesec{framed picture which is centered}
\begin{LTXexample}
\begin{mdframed}[userdefinedwidth=6cm,align=center,
                 linecolor=blue,linewidth=4pt]
\includegraphics[width=\linewidth]{donald-duck}
\end{mdframed}
\end{LTXexample}

\Examplesec{theorem with separate header and the help of TikZ (complex)}
\begin{mdexample}
\makeatletter
\newcounter{theo}[section]
\newcommand*\newmdframedtitleenv[1]{%
 \@afterindentfalse
  {\parindent \z@
    \setlength{\parfillskip}{\z@ plus 1fil}%
    \mdraggedtitle\nobreak%
    \makebox[\linewidth][l]{%
     \hspace*{-1\mdf@innerleftmargin@length}%
     \rlap{\color{white}%
       \hspace*{-1\mdf@middlelinewidth@length}%
       \rule[\mdf@middlelinewidth@length]%
            {\dimexpr\linewidth+1\mdf@innerleftmargin@length%
             +\mdf@innerrightmargin@length
             +2\mdf@middlelinewidth@length\relax}%
            {\dimexpr\ht\strutbox+.3333em\relax}%
     }%
     \rlap{\color{blue!20}%
       \rule{\dimexpr\linewidth+\mdf@innerleftmargin@length%
             +\mdf@innerrightmargin@length\relax}%
            {\mdf@middlelinewidth@length}}%
     \hspace*{-1\mdf@middlelinewidth@length}%
     \tikz[remember picture,baseline]%
       \node[,draw = none, text = black,fill = blue!20,]%
       {\mdf@frametitlefont\strut Theorem~\thetheo#1};\relax%
   }%
  \par\kern.5\baselineskip}%
 \@afterheading}
\newenvironment{theo}[1][]{%
  \let\mdframedtitleenv\newmdframedtitleenv%
   \stepcounter{theo}%
   \ifstrempty{#1}%
     {\mdfsetup{frametitle={\strut}}}%
     {\mdfsetup{frametitle={:~#1}}}%
   \begin{mdframed}[innertopmargin=0pt,linecolor=blue!20,%
                    linewidth=2pt,topline=false,]%
   }{\end{mdframed}}
\begin{theo}[Inhomogeneous Linear]
\ExampleText
\end{theo}

\begin{theo}
\ExampleText
\end{theo}
\end{mdexample}
\makeatletter
\newcounter{theo}[section]
\newcommand*\newmdframedtitleenv[1]{%
 \@afterindentfalse
  {\parindent \z@
    \setlength{\parfillskip}{\z@ plus 1fil}%
    \mdraggedtitle\nobreak%
    \makebox[\linewidth][l]{%
     \hspace*{-1\mdf@innerleftmargin@length}%
     \rlap{\color{white}%
       \hspace*{-1\mdf@middlelinewidth@length}%
       \rule[\mdf@middlelinewidth@length]%
            {\dimexpr\linewidth+1\mdf@innerleftmargin@length%
             +\mdf@innerrightmargin@length
             +2\mdf@middlelinewidth@length\relax}%
            {\dimexpr\ht\strutbox+.3333em\relax}%
     }%
     \rlap{\color{blue!20}%
       \rule{\dimexpr\linewidth+\mdf@innerleftmargin@length%
             +\mdf@innerrightmargin@length\relax}%
            {\mdf@middlelinewidth@length}}%
     \hspace*{-1\mdf@middlelinewidth@length}%
     \tikz[remember picture,baseline]%
       \node[,draw = none, text = black,fill = blue!20,]%
       {\mdf@frametitlefont\strut Theorem~\thetheo#1};\relax%
   }%
  \par\kern.5\baselineskip}%
 \@afterheading}
\newenvironment{theo}[1][]{%
  \let\mdframedtitleenv\newmdframedtitleenv%
   \stepcounter{theo}%
   \ifstrempty{#1}%
     {\mdfsetup{frametitle={\strut}}}%
     {\mdfsetup{frametitle={:~#1}}}%
   \begin{mdframed}[innertopmargin=0pt,linecolor=blue!20,%
                    linewidth=2pt,topline=false,]%
   }{\end{mdframed}}


\begin{theo}[Inhomogeneous Linear]
\ExampleText
\end{theo}

\begin{theo}
\ExampleText
\end{theo}

\clearpage
\Examplesec{hide only a part of a line}
The example below is inspired by the following post on StackExchange \href{http://tex.stackexchange.com/questions/24101/theorem-decorations-that-stay-with-theorem-environment}{Theorem decorations that stay with theorem environment}
\begin{LTXexample}
\makeatletter
\newlength{\interruptlength}
\setlength{\interruptlength}{2.5ex}
\newrobustcmd\overlaplines{%
 \appto\md@frame@leftline@single{%
  \llap{\color{white}%
   \rule[\dimexpr-\mdfboundingboxdepth%
         \ifbool{mdf@bottomline}{-\mdf@middlelinewidth@length}{}%
         +\interruptlength\relax]%
       {\mdf@middlelinewidth@length}%
       {\dimexpr\mdfboundingboxtotalheight%
        +\ifbool{mdf@bottomline}{\mdf@middlelinewidth@length}{0pt}
        +\ifbool{mdf@topline}{\mdf@middlelinewidth@length}{0pt}%
       -2\interruptlength\relax}%
  }% 
 }%
 \appto\md@frame@rightline@single{%
  \rlap{\color{white}%
    \hspace*{\mdfboundingboxwidth}%
    \hspace*{\mdf@innerrightmargin@length}%
    \rule[\dimexpr-\mdfboundingboxdepth%
          \ifbool{mdf@bottomline}{-\mdf@middlelinewidth@length}{}
          +\interruptlength\relax]%
         {\mdf@middlelinewidth@length}%
         {\dimexpr\mdfboundingboxtotalheight%
          +\ifbool{mdf@bottomline}{\mdf@middlelinewidth@length}{0pt}%
          +\ifbool{mdf@topline}{\mdf@middlelinewidth@length}{0pt}
          -2\interruptlength\relax}%
  }%
 }
}
\makeatother
\overlaplines


\begin{mdframed}[linecolor=blue,linewidth=8pt]
\ExampleText
\end{mdframed}
\end{LTXexample}
\end{document}
