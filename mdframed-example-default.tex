%%$Id: mdframed-examples.dtx 270 2011-12-09 12:19:09Z marco $
\setcounter{errorcontextlines}{999}
\documentclass[parskip=false,english,11pt]{ltxmdf}
\ltxmdfsetifoot  $Id: mdframed-examples.dtx 270 2011-12-09 12:19:09Z marco $

\usepackage{showexpl}
\lstset{style=lstltxmdf,explpreset={pos=b,rframe={}},}

\newcommand\Loadedframemethod{default}
\usepackage[framemethod=\Loadedframemethod]{mdframed}

\title{The \Pack{mdframed} package}
\subtitle{Examples for \Opt{framemethod=\Loadedframemethod}}
\author{\href{mailto:marco.daniel@mada-nada.de}{Marco Daniel}}
\version{\mdversion}
\introduction{In this document I collect various examples for \Opt{framemethod=\Loadedframemethod}.
Some presented examples are more or less exorbitant.}

\mdfsetup{skipabove=\topskip,skipbelow=\topskip}
\newrobustcmd\ExampleText{%
        An \textit{inhomogeneous linear} differential equation has the form
         \begin{align}
            L[v ] = f,
         \end{align}
        where $L$ is a linear differential operator, $v$ is
        the dependent variable, and $f$ is a given non-zero
        function of the independent variables alone.
}

\newcounter{examplecount}
\setcounter{examplecount}{0}
\renewcommand\thesubsection{}
\newcommand\Examplesec[1]{%
\stepcounter{examplecount}%
\subsection{Example~\arabic{examplecount}~--~#1\relax}%
}

\begin{document}
\maketitle
\section{Loading}
In the preamble only the package \Pack{mdframed} width the option \Opt{framemethod=\Loadedframemethod} is loaded. All other modifications will be done by \Cmd{mdfdefinestyle} or \Cmd{mdfsetup}.

{\large\color{red!50!black}
\NOTE Every \Cmd{global} inside the examples is necessary to work with the package \Pack{showexpl}.}

\section{Examples}
All examples have the following settings:

\begin{tltxmdfexample}
\mdfsetup{skipabove=\topskip,skipbelow=\topskip}
\newrobustcmd\ExampleText{%
An \textit{inhomogeneous linear} differential equation
has the form
\begin{align}
L[v ] = f,
\end{align}
where $L$ is a linear differential operator, $v$ is
the dependent variable, and $f$ is a given non-zero
function of the independent variables alone.
}
\end{tltxmdfexample}
\clearpage
\Examplesec{very simple}
\begin{LTXexample}
\global\mdfdefinestyle{exampledefault}{%
     linecolor=red,linewidth=3pt,%
     leftmargin=1cm,rightmargin=1cm
}
\begin{mdframed}[style=exampledefault]
\ExampleText
\end{mdframed}
\end{LTXexample}

\Examplesec{hidden line + frame title}
\begin{LTXexample}
\global\mdfapptodefinestyle{exampledefault}{%
 topline=false,rightline=true,bottomline=false}
\begin{mdframed}[style=exampledefault,frametitle={Inhomogeneous linear}]
\ExampleText
\end{mdframed}
\end{LTXexample}
\clearpage

\Examplesec{colored frame title}
\begin{LTXexample}

\global\mdfapptodefinestyle{exampledefault}{%
   rightline=true,innerleftmargin=10,innerrightmargin=10,
   frametitlerule=true,frametitlerulecolor=green,
   frametitlebackgroundcolor=yellow,
   frametitlerulewidth=2pt}
\begin{mdframed}[style=exampledefault,frametitle={Inhomogeneous linear}]
\ExampleText
\end{mdframed}
\end{LTXexample}

\Examplesec{framed picture which is centered}
\begin{LTXexample}
\begin{mdframed}[userdefinedwidth=6cm,align=center,
                 linecolor=blue,linewidth=4pt]
\includegraphics[width=\linewidth]{donald-duck}
\end{mdframed}
\end{LTXexample}
\clearpage
\Examplesec{theorem with separate header and the help of TikZ (complex)}
\begin{LTXexample}
\newcounter{theo}[section]
\newenvironment{theo}[1][]{%
 \stepcounter{theo}%
  \ifstrempty{#1}%
  {\mdfsetup{%
    frametitle={%
       \tikz[baseline=(current bounding box.east),outer sep=0pt]
        \node[anchor=east,rectangle,fill=blue!20]
        {\strut Theorem~\thetheo};}}
  }%
  {\mdfsetup{%
     frametitle={%
       \tikz[baseline=(current bounding box.east),outer sep=0pt]
        \node[anchor=east,rectangle,fill=blue!20]
        {\strut Theorem~\thetheo:~#1};}}%
   }%
   \mdfsetup{innertopmargin=10pt,linecolor=blue!20,%
             linewidth=2pt,topline=true,
             frametitleaboveskip=\dimexpr-\ht\strutbox\relax,}
   \begin{mdframed}[]\relax%
   }{\end{mdframed}}
\begin{theo}[Inhomogeneous Linear]
\ExampleText
\end{theo}

\begin{theo}
\ExampleText
\end{theo}
\end{LTXexample}

\clearpage
\Examplesec{hide only a part of a line}
The example below is inspired by the following post on StackExchange \href{http://tex.stackexchange.com/questions/24101/theorem-decorations-that-stay-with-theorem-environment}{Theorem decorations that stay with theorem environment}
\begin{LTXexample}
\makeatletter
\newlength{\interruptlength}
\setlength{\interruptlength}{2.5ex}
\newrobustcmd\overlaplines{%
 \appto\mdf@frame@leftline@single{%
   \llap{\color{white}%
      \rule[\dimexpr-\mdfboundingboxdepth+\interruptlength\relax]%
           {\mdf@middlelinewidth@length}%
           {\dimexpr\mdfboundingboxtotalheight%
            \ifbool{mdf@topline}{+\mdf@middlelinewidth@length}{}
            -2\interruptlength\relax}%
   }%
 }%
 \appto\mdf@frame@rightline@single{%
   \rlap{\color{white}%
      \hspace*{\mdfboundingboxwidth}%
      \hspace*{\mdf@innerrightmargin@length}%
      \rule[\dimexpr-\mdfboundingboxdepth%
            +\interruptlength\relax]%
           {\mdf@middlelinewidth@length}%
           {\dimexpr\mdfboundingboxtotalheight%
            +\ifbool{mdf@topline}{\mdf@middlelinewidth@length}{0pt}
            -2\interruptlength\relax}%
   }%
 }%
}
\makeatother
\overlaplines

\begin{mdframed}[linecolor=blue,linewidth=8pt]
\ExampleText
\end{mdframed}
\end{LTXexample}
\end{document}
 \endinput
