%Documenation of the package mdframed
%%$Id: mdframed-example-tikz.tex 220 2011-11-13 17:39:13Z marco $
\setcounter{errorcontextlines}{999}
\documentclass[parskip=false,english,11pt]{ltxmdf}
\svnInfo $Id: mdframed-example-tikz.tex 220 2011-11-13 17:39:13Z marco $
\usepackage{babel}
\usepackage[utf8]{inputenc}
\usepackage[T1]{fontenc}
\usepackage[T1,altbullet]{lucidabr}
\usepackage[scaled=0.82]{beramono}  

\usepackage[]{showexpl}
\lstset{style=lstmdframed,explpreset={pos=b,rframe={}},}

\newcommand\Loadedframemethod{TikZ}
\usepackage[framemethod=\Loadedframemethod]{mdframed}

\title{The \mdname package}
\subtitle{Examples for \mdoption{framemethod=\Loadedframemethod}}
\author{\href{mailto:marco.daniel@mada-nada.de}{Marco Daniel}}
\version{\mdversion}
\date{\svnToday}
\introduction{In this document I collect various examples for \mdoption{framemethod=\Loadedframemethod}. 
Some presented examples are more or less exorbitant.}


\mdfsetup{skipabove=\topskip,skipbelow=\topskip}
\newrobustcmd\ExampleText{%
        An \textit{inhomogeneous linear} differential equation has the form
         \begin{align}
            L[v ] = f,
         \end{align}
        where $L$ is a linear differential operator, $v$ is
        the dependent variable, and $f$ is a given non-zero 
        function of the independent variables alone.
}


\newcounter{examplecount}
\setcounter{examplecount}{0}
\renewcommand\thesubsection{}
\newcommand\Examplesec[1]{%
\stepcounter{examplecount}%
\subsection{Example~\arabic{examplecount}~--~#1\relax}%
}

\begin{document}
\maketitle
\section{Loading}
In the preamble only the package \mdname width the option \mdoption{framemethod=\Loadedframemethod} is loaded. All other modifications will be done by \mdcommand{mdfdefinestyle} or \mdcommand{mdfsetup}.

{\large\color{red!50!black}
\NOTE Every \mdcommand{global} inside the examples is necessary to work with the package \mdpack{showexpl}.}

\section{Examples}
All examples have the following settings:

\begin{mdexample}
\mdfsetup{skipabove=\topskip,skipbelow=\topskip}
\newrobustcmd\ExampleText{%
An \textit{inhomogeneous linear} differential equation
has the form
\begin{align}
L[v ] = f,
\end{align}
where $L$ is a linear differential operator, $v$ is
the dependent variable, and $f$ is a given non-zero 
function of the independent variables alone.
}
\end{mdexample}
\clearpage
\ExampleText{round corner}
\begin{LTXexample}
\global\mdfdefinestyle{exampledefault}{%
     outerlinewidth=5pt,innerlinewidth=0pt,
     outerlinecolor=red,roundcorner=5pt
}
\begin{mdframed}[style=exampledefault]
\ExampleText
\end{mdframed}
\end{LTXexample}

\Examplesec{hidden line + frame title}
\begin{LTXexample}
\global\mdfapptodefinestyle{exampledefault}{%
 topline=false,leftline=false,}
\begin{mdframed}[style=exampledefault,frametitle={Inhomogeneous linear}]
\ExampleText
\end{mdframed}
\end{LTXexample}

\clearpage
\Examplesec{Gimmick}
\begin{LTXexample}
\mdfsetup{splitbottomskip=0.8cm,splittopskip=0cm,
          innerrightmargin=2cm,innertopmargin=1cm,%
          innerlinewidth=2pt,outerlinewidth=2pt,
          middlelinewidth=10pt,backgroundcolor=red,
          linecolor=blue,middlelinecolor=gray,
          tikzsetting={draw=yellow,line width=3pt,%
                    dashed,%
                    dash pattern= on 10pt off 3pt},
          rightline=false,bottomline=false}
\begin{mdframed}
\ExampleText
\end{mdframed}
\end{LTXexample}


\Examplesec{complex example with TikZ}

\begin{mdexample}
\tikzstyle{titregris} =
          [draw=gray, thick, fill=white, shading = exersicetitle, %
           text=gray, rectangle, rounded corners,
           right,minimum height=.7cm]

\pgfdeclarehorizontalshading{exersicebackground}{100bp}
{color(0bp)=(green!40);
color(100bp)=(black!5)}

\pgfdeclarehorizontalshading{exersicetitle}{100bp}
{color(0bp)=(red!40);
color(100bp)=(black!5)}

\newcounter{exercise}
\renewcommand\theexercise{Exercise~n\arabic{exercise}}
\makeatletter
\def\mdf@@exercisepoints{}
\define@key{mdf}{exercisepoints}{%
    \def\mdf@@exercisepoints{#1}
}
\renewrobustcmd\mdfcreateextratikz{%
      \node[titregris,xshift=1cm] at (P-|O) %
           {~\mdf@frametitlefont{\theexercise}~};
      \ifdefempty{\mdf@@exercisepoints}%
      {}%
      {\node[titregris,left,xshift=-1cm] at (P)%
        {~\mdf@frametitlefont{\mdf@@exercisepoints points}~};}%
}
\makeatother

\mdfdefinestyle{exercisestyle}{%
  outerlinewidth=1pt,
  innerlinewidth=0pt,
  roundcorner=2pt,
  linecolor=gray,
  tikzsetting={shading = exersicebackground},
  innertopmargin=1.2\baselineskip,
  skipabove={\dimexpr0.5\baselineskip+\topskip\relax},
  needspace=3\baselineskip,
  frametitlefont=\sffamily\bfseries,
  settings={\global\stepcounter{exercise}},
  }

\begin{mdframed}[style=exercisestyle,]
\ExampleText
\end{mdframed}

\begin{mdframed}[style=exercisestyle,exercisepoints=10]
\ExampleText
\end{mdframed}
\end{mdexample}

\tikzstyle{titregris} =
          [draw=gray, thick, fill=white, shading = exersicetitle, %
           text=gray, rectangle, rounded corners,
           right,minimum height=.7cm]

\pgfdeclarehorizontalshading{exersicebackground}{100bp}
{color(0bp)=(green!40);
color(100bp)=(black!5)}

\pgfdeclarehorizontalshading{exersicetitle}{100bp}
{color(0bp)=(red!40);
color(100bp)=(black!5)}

\newcounter{exercise}
\renewcommand\theexercise{Exercise~n\arabic{exercise}}
\makeatletter
\def\mdf@@exercisepoints{}
\define@key{mdf}{exercisepoints}{%
    \def\mdf@@exercisepoints{#1}
}
\renewrobustcmd\mdfcreateextratikz{%
      \node[titregris,xshift=1cm] at (P-|O) {~\textbf{\theexercise}~};
      \ifdefempty{\mdf@@exercisepoints}%
      {}%
      {\node[titregris,left,xshift=-1cm] at (P)%
        {~\mdf@frametitlefont{\mdf@@exercisepoints points}~};}%
}
\makeatother

\mdfdefinestyle{exercisestyle}{%
  outerlinewidth=1pt,
  innerlinewidth=0pt,
  roundcorner=2pt,
  linecolor=gray,
  tikzsetting={shading = exersicebackground},
  innertopmargin=1.2\baselineskip,
  skipabove={\dimexpr0.5\baselineskip+\topskip\relax},
  needspace=3\baselineskip,
  frametitlefont=\sffamily\bfseries,
  settings={\global\stepcounter{exercise}},
  }

\begin{mdframed}[style=exercisestyle,]
\ExampleText
\end{mdframed}

\begin{mdframed}[style=exercisestyle,exercisepoints=10]
\ExampleText
\end{mdframed}


\end{document}